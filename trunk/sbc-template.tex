\documentclass[12pt]{article}
\usepackage{sbc-template}
\usepackage{graphicx,url}
\usepackage[brazil]{babel}
\usepackage[utf8x]{inputenc}
\usepackage{url}  % para usar URLs nas referencias
\sloppy


\title{Automação de Testes em Aplicações de BPM:\\ um Relato de Experiência}


\author{Jessica Lasch de Moura\inst{1},
Andrea Schwertner Charão\inst{1}}
\address{Núcleo de Ciência da Computação\\
Universidade Federal de Santa Maria -- UFSM
\email{\{jmoura, andrea\}@inf.ufsm.br}}



\begin{document}


\maketitle




\begin{resumo}
Este artigo aborda questões relacionadas ao teste automatizado de aplicações desenvolvidas com o apoio de sistemas de gestão de processos de negócio (Business Process Management Systems -- BPMS). Para isso, apresenta-se um relato de experiência de automação de testes de carga e de testes funcionais em uma aplicação de BPM, utilizando-se as ferramentas de código-aberto Apache JMeter e Selenium. Os resultados evidenciam limitações e oportunidades na automação de testes de aplicações de BPM.
\end{resumo}

\begin{abstract}
This article discusses the automated testing of applications developed with the support of Business Process Management Systems -- BPMS. We present an experience report on the automation of load tests and functional tests over a real BPM application, using the Apache JMeter and Selenium open-source tools. Results show the limitations and opportunities in test automation of BPM applications.
\end{abstract}




\section{Introdução}

Atualmente, a gestão de processos de negócio (\emph{Business Process Management} -- BPM) tem suscitado o interesse de empresas e da comunidade científica, tanto por seus benefícios como por seus desafios. Designa-se por BPM o conjunto de conceitos, métodos e técnicas para suportar a modelagem, administração, configuração e análise de processos de negócio~\cite{weske, aalst2013survey}. Associados a isso, surgiram os sistemas BPM (\emph{Business Process Management Systems} ou \emph{Suites} -- BPMS), que são ferramentas de software para apoio ao ciclo de vida da gestão de processos de negócio. Tais ferramentas, quando bem aplicadas, têm o potencial de alavancar aumentos de produtividade e redução de custos nos mais variados tipos de organizações. 

Dentre os diversos BPMS disponíveis atualmente, é comum encontrar ferramentas com suporte a modelagem, configuração e execução de processos de negócio. Por outro lado, tarefas relacionadas à simulação, monitoramento, verificação e testes ainda são consideradas um desafio nesta área~\cite{aalst2013survey}. Em particular, o teste automatizado de aplicações de BPM é pouco abordado, tanto pela comunidade da área de BPM~\cite{aalst2013survey} como da área de testes de software~\cite{graham2012experiences}. Diante disso, estima-se que muitas organizações se limitem a testes manuais em suas aplicações de BPM. No entanto, a falta de automação nos testes pode levar a vários problemas durante a implementação e execução de processos de negócio, como baixa aderência aos requisitos, maior esforço dos desenvolvedores, desperdício de tempo e aumento do risco de duplicação de esforços e de erro humano. 

Nesse contexto, o presente artigo relata uma experiência de automação de testes em uma aplicação de BPM desenvolvida para agilizar um processo em uma instituição pública de ensino. Todas as etapas do desenvolvimento da aplicação, utilizando o BPMS Bonita Open Solution (BOS), foram apresentadas em um trabalho precedente~\cite{sbsi2013}. 


\section{BPM e Testes}

Em muitos casos, o termo BPM pode ser usado com significados diferentes~\cite{acmxrds2009}, às vezes com mais ênfase em tecnologia (software) e outras vezes mais associado a gestão. Mesmo assim, a área tem convergido sobre o ciclo de vida de aplicações de BPM, que envolve as atividades de análise, modelagem, execução, monitoramento e otimização~\cite{ABPMP}. 

%Os sistemas de BPM (BPMS) têm se afirmado como ferramentas essenciais para suporte a atividades desse ciclo de vida. Atualmente, pode-se dizer que um típico BPMS oferece recursos para definição e modelagem gráfica de processos, controle da execução e monitoramento de atividades dos processos. Alguns exemplos de BPMS que se destacam neste cenário são IBM Websphere ~\cite{WEBSPHERE}, Oracle BPM Suite~\cite{ORACLEBPM}, Intalio~\cite{INTALIO}, Bizagi~\cite{BIZAGI}, TIBCO BPM~\cite{TIBCOBPM}, Activiti~\cite{ACTIVITI} e Bonita Open Solution ~\cite{BONITASOFT}.
% Referencias acima foram retiradas por ocuparem muito espaco no final. Podem ser substituidas por notas de rodape, caso haja espaco.

Os sistemas de BPM (BPMS) têm se afirmado como ferramentas essenciais para suporte a atividades desse ciclo de vida. Atualmente, pode-se dizer que um típico BPMS oferece recursos para definição e modelagem gráfica de processos, controle da execução e monitoramento de atividades dos processos. Alguns exemplos de BPMS que se destacam neste cenário são IBM Websphere, Oracle BPM Suite, Intalio, Bizagi, TIBCO BPM, Activiti e Bonita Open Solution.

Nota-se que a preocupação com testes não fica evidente na ferramentas BPMS. De fato, analisando-se o material promocional e a documentação publicamente disponível sobre os principais BPMS, observa-se uma ênfase em etapas de modelagem e execução. Visivelmente, tais recursos são um diferencial no desenvolvimento de aplicações de BPM, em comparação ao desenvolvimento de software em geral. No entanto, aplicações de BPM também estão sujeitas a defeitos e, por isso, podem se beneficiar de avanços na área de testes de software.

%Em engenharia de software, o teste é tradicionalmente considerado uma prática importante~\cite{pressman, swebok}. Há uma vasta terminologia relacionada a testes de software, classificando-os de acordo com diferentes critérios (objetivos, técnicas, entre outros). Por exemplo, os testes podem abranger o sistema inteiro (teste de sistema), alguns componentes (teste de integração) ou unidades isoladas (teste unitários). Alguns autores também distinguem testes funcionais, que avaliam o comportamento do software frente a seus requisitos, dos testes ditos não-funcionais, que verificam atributos relacionados aos requisitos não-funcionais do software, como por exemplo desempenho e usabilidade~\cite{desikan2006software}. Técnicas de teste podem variar de acordo com a natureza da aplicação~\cite{swebok}, por exemplo: orientada a objetos, baseada na Web, com interface gráfica, etc.
% citar livro do Cem Kaner

Em testes de software, há muitas tarefas que podem ser trabalhosas e propensas a erros quando realizadas manualmente. Por este fato, vários autores relatam a importância dos testes automatizados em ambientes de desenvolvimento~\cite{sbqs2013}. Com a evolução das áreas de qualidade e teste de software, foi surgindo uma variedade de soluções para automação de testes. Ferramentas para testes unitários, por exemplo, são numerosas e costumam se integrar aos ambientes de desenvolvimento~\cite{unittesting}. Outro exemplo é o teste de aplicações baseadas em interfaces gráficas ou Web~\cite{webtesting}.

%http://en.wikipedia.org/wiki/List_of_GUI_testing_tools

%Pressman define quatro tipos de teste de software: teste de unidade, teste de integração, teste de validação, teste de sistema. Teste de unidade concentra-se em cada unidade do software, de acordo com o que é implementado no código fonte, utiliza as técnicas de teste de caixa branca e caixa preta. Teste de integração concentra-se no projeto e na construção da arquitetura de software, utilizando principalmente as técnicas de teste de caixa preta. No teste de validação os requisitos estabelecidos como parte da análise de requisitos de software são validados em relação ao software que foi construído. Por último, no teste de sistema, o software e outros elementos do sistema são testados como um todo. Teste de segurança e recuperação \cite{pressman1995engenharia}. O teste de sistema pode ser dividido em uma série de diferentes testes, cujo objetivo principal é por completamente à prova o sistema, dentre estes sub-tipos de testes está o teste de carga.

Assumindo que aplicações de BPM podem ser tratadas como software em geral, é possível testá-las sob diferentes aspectos, por meio de tipos de testes já consagrados em engenharia de software, como por exemplo testes funcionais do tipo caixa-preta ou teste de carga. Sob esta ótica, pode-se empregar ferramentas de automação de testes alinhadas com cada tipo de teste. No entanto, a adoção esta abordagem pode ter limitações e dificuldades, pois não leva explicitamente em conta o ciclo de vida de aplicações de BPM.

%Há alguns anos, o termo \emph{Business Process Testing -- BPT} vem sendo empregado para designar testes de processos de negócio. Não há uma caracterização clara deste tipo de teste, sendo que em alguns contextos o termo refere-se a testes automatizados reusáveis criados por especialistas do domínio~\cite{hp}, portanto aderentes aos processos de negócio. Em outros contextos, o termo relaciona-se a automação de testes de processos implementados em arquiteturas orientadas a serviços (\emph{Service Oriented Architectures} -- SOA)~\cite{soatest2008, bpeltest2008}. Testes deste tipo possuem uma relação com BPM e são uma especialização de testes de software em geral. No entanto, pode-se dizer que essa relação com BPM é fraca, pois é principalmente focada na etapa de execução dos processos. Além disso, não costumam ser soluções integradas em sistemas BPM.


% sistemas BPMS: simulação, monitoramento...
%http://www8.hp.com/us/en/software-solutions/software.html?compURI=1174789#.UvTvdfjei1H


%Aspectos comuns a qualquer tipo de software
%- verificação (conformidade, corretude, ...)


%Aspectos particulares
%- verificação dos modelos BPMN
%- caminhos críticos e gargalos (simulação)


%Os desafios da homologação de Processos Automatizados
%http://blog.iprocess.com.br/2013/04/os-desafios-da-homologacao-de-processos-automatizados/


\section{Processo alvo dos testes e ?}
Os testes realizados neste trabalho referem-se a um processo realizado com frequência em instituições de ensino superior, que é a apreciação de Atividades Complementares de Graduação (ACGs), ou seja, atividades que formam a parte flexível do currículo dos graduandos (participação em palestras, eventos, projetos, etc.). Em um trabalho anterior ~\cite{sbsi2013}, esse processo foi modelado e implantado utilizando a ferramenta Bonita Open Solution. Como mostra a Figura \ref{fig:diagrama}. É um processo complexo que permite analisar diversas funcionalidades dos BPMS e, justamente por já ter sido alvo de um trabalho, é um processo no qual já se tem uma grande experiência.

\begin{figure}[ht]
\centering
\includegraphics[width=.99\textwidth]{imagens/processo.png}
\caption{Diagrama da modelagem do processo}
\label{fig:diagrama}
\end{figure}

Devido ao tempo limitado para a conclusão deste trabalho, decidiu-se por escolher duas ferramentas para o estudo. Primeiramente, foi escolhida a ferramenta Bonita Open Solution, devido a esta já ter sido usada em um trabalho anterior ~\cite{sbsi2013} e, por isso, tem se uma vasta experiência nesta ferramenta.

O Bonita Open Solution (BOS) é uma ferramenta distribuída sob uma licença de software livre, desenvolvida em Java, pela empresa BonitaSoft~\cite{BONITASOFT}. A ferramenta BOS oferece componentes tanto para a modelagem como para a implementação e transformação de processos. A modelagem e customização de processos é realizada através do Bonita Studio, um componente com interface gráfica tipo desktop que agrupa ferramentas de desenvolvimento. Também é possível agregar funcionalidades aos processos através de conectores que podem, inclusive, receber personalizações feitas através de códigos.

Analisando as ferramentas disponíveis e levando em conta os trabalhos e livros publicados ~\cite{rademakers2012activiti} que abordam o uso da ferramenta, o segundo software escolhido foi o Activiti. O Activiti ~\cite{ACTIVITI} é um BPMS de código aberto, distribuído sob a uma licença Apache, criado em Java e usa  BPMN 2.0 para a modelagem dos processos, ele pode ser executado em qualquer plataforma, servidor, cluster ou na nuvem. A ferramenta Activiti oferece componentes distintos para modelagem, implementação e execução de processos. A modelagem e customização dos processos é feita no Activiti Designer que é um plugin para a plataforma Eclipse, o que torna o ambiente de criação fácil de ser usado, a execução é feita no componente chamado Activiti Explorer.
%tabelinha de caminhos? monografia metodologia.tex

\section{Descrição e execução dos Testes}

No planejamento dos testes automatizados, priorizou-se o teste de  etapas que de fato revelaram problemas durante a operação. Com isso, buscou-se verificar se os problemas poderiam ser facilmente identificados antes de colocar-se a aplicação em produção. Os testes escolhidos foram: (a) testes de carga, que são um tipo de teste de desempenho, visando avaliar o comportamento do sistema frente a um grande número de solicitações e (b) testes funcionais, a fim de verificar as saídas do sistema produzidas a partir de entradas pré-definidas. \textbf{Nenhum destes tipos de teste possui suporte no BPMS utilizado} (Bonita Open Solution), que inclui somente funcionalidades limitadas de simulação e depuração de execução dos processos. Assim, realizou-se um levantamento de ferramentas de teste disponíveis e selecionou-se as mais promissoras, antes de partir-se para o detalhamento e execução dos testes.

%jmeter
%-pq escolheu,
%processo de teste nas duas
%problemas com as duas ferramentas
%resultados

%selenium+cucumber
%-pq escolheu
%processo do teste nas duas
%problemas com as duas
%resultados

%tabela com informações gerais

\begin{table}
\centering
\begin{tabular}{|p{2cm}|p{7cm}|p{7cm}|}
  & Teste de Carga & Teste Funcional  \\\hline
Descrição & 
\begin{itemize}
\item{teste de desempenho, visa avaliar o comportamento do sistema frente a um grande número de solicitações;}
\item{durante a implantação do processo testado foi a falha do sistema perante um grande acesso de usuários em um dia específico.}
\end{itemize}
&
\begin{itemize}
\item{Teste que verifica as saídas do sistema produzidas a
partir de entradas pré-definidas;}
\item{Durante a utilização do processo
dentro da instituição, ocorreram alguns erros como: campo sem suporte a caracteres especiais,
erro devido a nome de arquivos muito grandes, entre outros.}
\end{itemize}
 \\\hline
Ferramentas &
\begin{itemize}
 \item{JMeter}
 \begin{itemize}
  \item{Ferramenta independente de plataformas;}
  \item{Permite a criação de diversos usuários virtuais  para a execução dos testes.}
  \end{itemize}
\end{itemize}
 & 
\begin{itemize}
\item{Selenium:Um de seus componenetes é um plugin para o navegador Firefox, capaz de registrar e reproduzir interações do usuário com o navegador, assim permitindo criar scripts de teste, sem escrita de código;}
\item{Cucumber JVM:Ferramenta que executa descrições de teste, em texto simples, como testes automatizados.}
\end{itemize} 
 \\\hline
 Processo de teste &
 \begin{enumerate} 
\item{Capturar requisições HTTP;}
\item{Exportar requisições (Formato .jrxml para JMeter);}
\item{Configurar Plano de teste (script);}
\item{Executar teste (JMeter).}
     \end{enumerate}  
 &   
\begin{enumerate}   
 \item{Captura da execução no navegador  (Selenium);}
 \item{Exportar código gerado  (Selenium);}
       \item{Criar cenário de teste (Cucumber);}
       \item{Criar as definições do passos de execução (Cucumber);}
       \item{Criar métodos (Java);}
       \item{Executar o teste (Selenium).}
       \end{enumerate}  
 \\\hlines
\end{tabular}
\caption{Levantamento sobre as ferramentas: Informações gerais}
\label{tab:bpms1}
\end{table}



\begin{table}
\centering
\begin{tabular}{|p{2cm}|p{7cm}|p{7cm}|}
  & Teste de Carga & Teste Funcional  \\\hline
Informações gerais sobre os testes & Foram executados testes simulando 1,50,100 e 200 usuários virtuais; Os testes foram executados na etapa de login e na execução da primeira tarefa do processo;
 &  \\\hline
BPMS Bonita Open Solution (BOS) & \begin{itemize}\item{Houveram problemas com a execução de requisições que utilizavam a tecnologia Google Web Toolkit (GWT), foi resolvido atráves de estudo das requisições e utilização de ferramentas auxiliares para a captura \cite{BLAZEMETER};}
\item{Em sistemas BPM as tarefas de um processo são interligadas e/ou dependentes entre si, no BOS existe uma chave que identifica cada execução do processo como única, e é criada no momento em que o usuário inicia o processo. Essa particularidade causou no problemas na execução dos testes e, para contorna-los, foi necessário estudar a fundo as requisições e o processo e utilizar opções da ferramenta JMeter bem como scripts para adaptar as requisições.}\end{itemize}
 & \begin{itemize} 
\item{A captura da execução funciona bem para a estrutura das páginas web da aplicação;}
\item{Ao executar o teste, alguns erros ocorrem devido a ferramenta Selenium “Buscar” elementos que ainda não foram carregados na página. Isto pode ser resolvido a partir da inserção de pequenos scripts nos métodos.}
 \end{itemize} \\\hline
BPMS Activiti & Ocorreram os mesmos erros relativos à dependência entre as tarefas do processo, as chaves identificadoras foram encontradas, no entanto, foi impossível identificar em que requisição as mesmas eram geradas, não havia uma requisição
cujo o retorno (resposta do servidor) contivesse as chaves utilizadas. Esta situação leva a crer
que a geração das chaves identificadoras é feita internamente pelo BPMS, ou seja, não em uma
requisição HTTP e, por sequência, esta não pode ser capturada e importada no Jmeter. & 
Alguns elementos das páginas que compõem a aplicação não são capturados pelo Selenium, acredita-se que esse problema ocorra devido a estrutura da páginas web que pode conter elementos com os quais o Selenium não trabalhe bem como divs,frames e
scrips, por exemplo. Para contornar esse problema, foi necessário estudar a estrutura das páginas que web, localizar os elementos faltantes e então adicionar o código para acessá-los nos respectivos métodos.\\\hline
\end{tabular}
\caption{Levantamento sobre as ferramentas: Informações gerais}
\label{tab:bpms2}
\end{table}


\section{Trabalhos Relacionados}

%artigos sbqs 2014
%artigos sbqs 2013

%O teste de software voltado especificamente a aplicações de BPM é um assunto que pode ser considerado ainda em aberto. Atualmente, muitos trabalhos de pesquisa têm focado em etapas de monitoramento e otimização em BPM~\cite{Gambini:2011:AEC:2040283.2040300, Liu:2011:BAM:2040283.2040307, deLeoni:2012:AEL:2413516.2413525, Ramezani:2012:DIM:2413516.2413545}, que são etapas dedicadas a identificar e corrigir problemas. Embora exista alguma relação com testes em BPM, tratam-se geralmente de abordagens mais voltadas a aspectos de gestão, não de software . Por outro lado, um assunto que tem sido bastante abordado é o teste automatizado de aplicações orientadas a serviços Web~\cite{soatest2008}. Embora se trate de um nicho de teste de software, e mesmo que haja uma relação entre BPMS e serviços Web, acredita-se que esta abordagem não abarque toda a problemática do teste de aplicações de BPM. Na aplicação alvo deste trabalho, em particular, a abordagem orientada a serviços Web não poderia ser usada, pois o BPMS empregado baseia-se numa arquitetura que não expõe seus serviços.

%No que diz respeito a relatos de experiência e estudos de caso, costuma haver espaço para isso em conferências internacionais sobre BPM e engenharia de software. Conforme van der Aalst (2013), em uma análise de várias edições da International Conference on Business Process Management, há muitos artigos que descrevem esforços de implementação e estudos de caso. No entanto, vários deles envolvem software que não é disponível ao leitor ou casos que são deliberadamente vagos~\cite{aalst2013survey}. No Brasil, conferências como o Simpósio Brasileiro de Sistemas de Informação e o Simpósio Brasileiro de Engenharia de Software incluem BPM e testes de software entre seus tópicos de interesse mas, até onde foi possível verificar, ainda não foram publicados trabalhos associando esses dois tópicos.

%Embora o termo ``teste'' não seja frequente na literatura sobre BPM, o ciclo de vida de aplicações de BPM inclui as etapas de monitoramento e otimização, que se dedicam a identificar e corrigir problemas~\cite{weske}. Tal visão do ciclo de vida é comumente voltada a aspectos de gestão, não de tecnologia (software). Mesmo assim, acreditamos que o teste de software relacione-se particularmente com essas etapas e, de forma geral, possa contribuir significativamente para o sucesso de aplicações de BPM.

%O artigo 'Um estudo sobre testes de desempenho com aplicação prática utilizando a ferramenta JMeter' (referencia) descreve o resultado de um estudo sobre testes de desempenho aplicado teste de desempenho aplicado em uma arquitetura e-commerce hipotética utilizando o JMeter. Diferentemente do presente trabalho, este artigo não possui usuários e um sistema real para ser analisado, sendo seu o objetivo analisar os dados para o desenvolvimento de um aplicativo, não para a melhoria do sistema.

%selenium e JMeter
%No trabalho “A Test Automation Framework Based on Web” \cite{wang2012test} é relatada a criação de um framework para teste automatizado de aplicações web, utilizando as ferramentas Selenium e JMeter. Os resultados do trabalho demonstram que as ferramentas ajudam, bem como o framework, ajudam a melhorar a qualidade do software e a aumentar a eficiência do desenvolvimento.

%testes com Selenium;
%O artigo “Automating functional tests using Selenium”\cite{holmes2006automating} é um relato de experiência utilizando o Selenium para testes automatizados e descreve as dificuldades e aprendizados com esta ferramenta, como tempo para escrever os scripts de teste bem como integração. Entretanto, os testes são baseados em aplicações Web simples e não em BPM. De fato não foram encontrados artigos que unissem BPM e a ferramenta Selenium para testar processos.

%trabalhos mais teoricos sobre teste de BPM (possivelmente citar na Seção 2)
%No white-paper “Performance Testing of Business Process Management (BPM) aplications using JMeter” \cite{} é um artigo teórico que defende a importância do teste de performance nas aplicações baseadas em BPM, para evitar a falha dos processos no ambiente de produção, bem como defende o uso do JMeter para implementar os testes levantando várias justificativas, dentre elas o fato de ser uma ferramenta gratuita que tem tantas funcionalidades quanto ferramentas pagas. Diferente do nosso trabalho, este não possui um ambiente em produção ou um processo real, com problemas reais, para efetuar o teste com o JMeter. O trabalho apenas trata da importância dos testes e sobre a potencialidade do JMeter.

\section{Conclusão}
Neste trabalho, foram conduzidas experiências de teste de software em uma aplicação real de BPM, que encontra-se implantada em uma instituição pública de ensino. Na delimitação o trabalho, focou-se em dois tipos de teste: testes de carga e testes funcionais. Na indisponibilidade de ferramentas específicas para o teste de aplicações de BPM, buscou-se selecionar ferramentas consagradas na área de teste de software.

As experiências de testes de carga mostraram que uma ferramenta típica para este propósito (Apache JMeter), mesmo não sendo integrada ao BPMS, tornou possível atingir os objetivos dos testes. Em particular, conseguiu-se determinar um nível de carga em que o tempo de resposta seria muito alto para os usuários da aplicação de BPM.

Por outro lado, os testes funcionais, mesmo utilizando uma ferramenta popular (Selenium), não atingiram plenamente os objetivos. Esses resultados sugerem que a automação deste tipo de teste não seja uma tarefa trivial em aplicações de BPM, principalmente pela falta de ferramentas adequadas e alinhadas com o ciclo de vida de BPM.

Embora as experiências não tenham sido exaustivas, acredita-se que constituam uma contribuição para profissionais e pesquisadores interessados dedicados aos tópicos de BPM e teste de software. Como trabalhos futuros, pode-se destacar a exploração de ferramentas de geração de casos de teste, na hipótese de que possam ajudar a alinhar os testes com as saídas e entradas do processo. Outra via que merece ser explorada são os testes de regressão, para auxiliar a encontrar possíveis problemas após alterações no processo, que podem ser frequentes dependendo do caso.

\bibliographystyle{sbc}
\bibliography{sbc-template}


\end{document}





